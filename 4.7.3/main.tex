
\documentclass[a4paper, 12pt]{article}
\usepackage[a4paper,top=1.5cm, bottom=1.5cm, left=1cm, right=1cm]{geometry}
\usepackage{cmap}
\usepackage{mathtext}
\usepackage[T2A]{fontenc}
\usepackage[utf8]{inputenc}
\usepackage[english,russian]{babel}
\usepackage{graphicx}
\usepackage{tabularx}
\usepackage{float}
\usepackage{longtable}
\usepackage{hyperref}
\hypersetup{colorlinks=true,urlcolor=blue}
\usepackage[rgb]{xcolor}
\usepackage{amsmath, amsfonts, amssymb, amsthm, mathtools}
\usepackage{euscript}
\usepackage{mathrsfs}
\usepackage{enumerate}
\usepackage{caption}
\usepackage{enumerate}
\mathtoolsset{showonlyrefs=true}
\usepackage{graphicx}
\usepackage{caption}
\usepackage{subcaption}
\usepackage{amsthm}

\title{\textbf{Длинные линии (3.7.3)}}
\author{Манро Эйден}
\date{}

\begin{document}

\maketitle

\noindent \textbf{Цель работы:} Ознакомится и проверить на практике теорию распространения
электрических сигналов вдоль длинной линии; измерить амплитудо- и фазово-частотные
характеристики коаксиальной линии; определить погонные характеристики такой
линии; на примере модели длинной линии изучить вопрос распределения амплитуды
колебаний сигнала по длине линии.

\bigskip

\noindent \textbf{В работе используются:} осциллограф АКТАКОМ ADS-6142H; генератора АКИП 3420/1; бухта с коаксиальным кабелем pk 50-4-11;
	схематический блок "модель длинной линии"; магазин сопротивления Р33, соединительные провода.

\end{document}
