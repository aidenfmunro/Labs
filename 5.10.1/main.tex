\documentclass[a4paper,12pt]{article} % тип документа

% Поля страниц
\usepackage[left=2.5cm,right=2.5cm, top=2cm,bottom=2cm,bindingoffset=0cm]{geometry}
    
%Пакет дял таблиц   
\usepackage{multirow} 
    
%Отступ после заголовка    
\usepackage{indentfirst}


% Рисунки
\usepackage{subcaption,floatrow,graphicx,calc}
\usepackage{wrapfig}

% Создаёем новый разделитель
\DeclareFloatSeparators{mysep}{\hspace{1cm}}

% Ссылки?
\usepackage{hyperref}
\usepackage[rgb]{xcolor}
\hypersetup{				% Гиперссылки
    colorlinks=true,       	% false: ссылки в рамках
	urlcolor=blue          % на URL
}


%  Русский язык
\usepackage[T2A]{fontenc}			% кодировка
\usepackage[utf8]{inputenc}			% кодировка исходного текста
\usepackage[english,russian]{babel}	% локализация и переносы

\usepackage{siunitx}


% Математика
\usepackage{amsmath,amsfonts,amssymb,amsthm,mathtools, mathrsfs, wasysym}

\title{\textbf{Электронный парамагнитный резонанс} (5.1.1)}
\author{Манро Эйден Б01-303б}
\date{}

\begin{document}

\maketitle

\noindent \textbf{Цель работы}: Исследуется электронный парамагнитный резонанс в молекуле ДФПГ, определяется $g$-фактор электрона, измеряется ширина ЭПР.

\begin{center}
\section*{Теоретическая часть}
\end{center}

Внешнее магнитное поле с индукцией $B$ приводит к расщеплению исходного энергетического уровня электрона на два подуровня. Разность энергий между ними выражается соотношением
\begin{equation}
    \label{eq:dE}
    \Delta E = E_2 - E_1 = 2\mu B,
\end{equation}

где $\mu$ обозначает модуль проекции магнитного момента на направление магнитного поля.

Между этими состояниями возможны квантовые переходы. Они могут быть индуцированы приложенным переменным электромагнитным полем высокой частоты, если его направление и частота удовлетворяют условиям резонанса.

Резонансная частота $\omega_0$ определяется условием

\begin{equation}
    \label{eq:resonans_omega}
    \hbar \omega_0 = \Delta E.
\end{equation}

При переходе электрона на более высокий уровень он поглощает квант энергии электромагнитного излучения, а при обратном переходе испускает такой же квант. Явление возбуждения этих переходов внешним полем с частотой, задаваемой формулой (\ref{eq:resonans_omega}), называется электронным парамагнитным резонансом (ЭПР).

В данной работе ставится задача зарегистрировать сигнал ЭПР на образце кристаллического дифенилпикрилгидразила (ДФПГ) и вычислить $g$-фактор электрона. Связь между магнитным моментом $\mu$ и механическим моментом $\mathbf{M}$ задаётся гиромагнитным отношением $\gamma$:

\begin{equation}
    \label{eq:gyromagnit}
    \mu = \gamma M.
\end{equation}

Если магнитный момент выражается в магнитонах Бора, а механический — в единицах $\hbar$, то это соотношение принимает вид
\begin{equation}
    \label{eq:def_g}
    \frac{\mu}{\mu_\text{Б}} = \frac{M}{\hbar}.
\end{equation}

Комбинируя выражения (\ref{eq:dE})–(\ref{eq:def_g}), получаем итоговую формулу для определения $g$-фактора через параметры, доступные эксперименту:
\begin{equation}
    \label{eq:g_is}
    \tag{$\star$}
    g = \frac{\hbar \omega_0}{\mu_\text{Б} B}.
\end{equation}

\newpage

\begin{center}
    \section*{Экспериментальная установка}
\end{center}

Образец (порошок ДФПГ) в стеклянной ампуле помещяется внутрь катушкииндуктивнсоти входящей в состав колебательного контура. Входящий в состав контура конденсатор состоит из двух платсин, разделенных воздушным зазором, одна из пластин может перемещаться поворотом штока. Колебания в контуре возбуждаются антенной, соединённой с генератором частоты (ВЧ) Г4-116. Амплитуда колебаний поля в катушке индуктивности измеряется по наводимой в петле связи ЭДС индукции. Высокочастотные колебания ЭДС индукции в приёмном контуре детектируются диодом, измеряемая при помощи осциллографа низкочастотная огибающая этого сигнала пропорциональна квадрату амплитуды колебаний поля в катушке.

\begin{figure}[h!]
    \centering
        \caption{Схема установки}
        \label{fig:equip}
        \includegraphics[scale=0.17]{equip.png}
\end{figure}

Постоянной магнитное поле создаётся пропусканием тока от источника постоянного тока через основные катушки. При этом при помощи вольтметра измеряется падение напряжения на резисторе в цепи основных катушек. Переменное поле небольшой амплитуды создаётся подачей на модуляционные катушки напряжения с регулируемого трансформатора ЛАТР. Для измерения амплитуды колебаний переменного поля используется пробная катушка известной геометрии, подключенная к вольтметру.



\newpage

\begin{center}
\section*{Ход работы}
\end{center}
    
\subsection*{Резонанс}

Настроим генератор на частоту колебательного конутра. Получаем резонансную частоту:
\begin{equation*}
    f_0 = (163 \pm 1) \ \text{Мгц}.
\end{equation*}

Подберем величину постоянного магнитного поля в катушках так, чтобы наблюдался сигнал резонанского поглощения. Для этого подадим на катушки достаточное напряжение.

Для более точной настройки и определения ширины линии резонасного поглощения будем наблюдать сигнал в $XY$-режиме. Запишем значение напряжения на резисторе в цепи основных катушек:
\begin{equation*}
    U_0 = (129 \pm 1) \ \text{мВ}.
\end{equation*}

\subsection*{Ширина линии поглощения}

Определим ширину линии ЭПР (полуширина на на полувысоте линии резонасного поглощения):
\begin{equation*}
    \Delta B = \frac{A_{1/2}}{A_{\text{полн}}}B_\text{мод},
\end{equation*}
где $A_\text{полн}$ -- полный размах модулирующего поля, $A_{1/2}$ -- ширина кривой на полувысоте, $B_\text{мод}$ -- амплитуда модулирующего поля
\begin{equation*}
        A_\text{полн} = (6.0 \pm 0.1) \ \text{дел}, \ A_{1/2} = (1.0 \pm 0.1) \ \text{дел} \\
\end{equation*}

$$ \Delta B = (0.13 \pm 0.02) \ \text{мТл}} $$

\subsection*{Калибровка основной катушки}

Определим связь между падением напряжения на резисторе в цепи основных катушек и магнитным полем в центре магнита. Поле в центре будем измерять, поднося пробную катушку к основным с двух сторон - спереди и сзади. 

Методом наименьших квадратов найдем коэффициент пропорциональности между напряжением на основных катушках и напряжением на пробной катушке:

\begin{equation*}
    k = 0.113 \pm 0.001
\end{equation*}

\begin{figure}[h!]
    \begin{center}
    \includegraphics[width = 1\textwidth]{probe_plot.png}
    \caption{График зависимости ЭДС индукции в пробной катушке от падения напряжения на резисторе в цепи питания катушки}
    \end{center}
\end{figure}


Рассчитав поле, создаваемое основными катушками,

\begin{equation*}
    B_0 = \frac{k U_0}{\omega_0 N S} = (5.1 \pm 0.1) \; \text{мТл}.
\end{equation*}

Найдем $g$-фактор электрона:

\begin{equation*}
    g = \frac{hf_0}{\mu_BB_0} = 2.0 \pm 0.2
\end{equation*}

\begin{center}
\section*{Вывод}
\end{center}
    
В ходе лабораторной работы был экспериментально исследован эффект электронного парамагнитного резонанса (ЭПР). На кристалле ДФПГ удалось зарегистрировать резонансный сигнал, возникающий при совпадении частоты внешнего электромагнитного поля с собственной частотой перехода электрона между магнитными подуровнями.

Эксперимент подтвердил, что расщепление энергетических уровней электрона в магнитном поле описывается формулой

$$
\Delta E = 2\mu B,
$$

а условие резонанса удовлетворяет соотношению

$$
\hbar \omega_0 = \Delta E.
$$

По результатам измерений была определена величина $g$-фактора электрона. Полученное значение оказалось близким к табличному ($g \approx 2.0$), что свидетельствует о правильности методики и точности эксперимента.

Практическая часть работы показала:

1. Электронный парамагнитный резонанс является надёжным методом изучения магнитных свойств веществ.
2. Кристалл ДФПГ является удобным эталонным образцом для регистрации ЭПР-сигнала благодаря наличию неспаренных электронов.
3. Используемая установка позволяет определить резонансное магнитное поле и частоту возбуждающего сигнала, на основе которых вычисляется $g$-фактор.

Таким образом, в работе была достигнута основная цель: получен сигнал ЭПР и экспериментально подтверждена связь между частотой резонансного перехода и величиной магнитного поля. Это позволило рассчитать значение гиромагнитного отношения электрона и убедиться в его согласии с теоретическими данными.


	





\end{document}